\documentclass[10pt, a4paper]{article}
\usepackage[parfill]{parskip}
\usepackage[margin=0.9in]{geometry}
\usepackage{url}
\usepackage{listings}

\lstdefinestyle{snippet}{
    numbers = left,
    basicstyle = \ttfamily\footnotesize,
    frame = single,
    breaklines = true,
}

\setlength\parindent{0pt}

\begin{document}
\title{CS547 Advanced Topics in Computer Science\\
\large{Assignment 03 - Multi-Objective Optimisation: The Next Release Problem}}
\author{Aidan O'Grady - 201218150}
\date{}
\maketitle

\section{Overview}
\label{sec:overview}
The task given revolved around the prioritisation of a set of requirements and
a set of customers, and to compare how to achieve an optimal solution utilising
both multi-objective and single-objective optimisation, using a random search
optimisation as a baseline for comparison, achieved with the Opt4J framework.

Known as the Next Release Problem, a software developer has a list of
requirements to be implemented for the next release, each with a certain cost
and dependencies on other requirements. In addition to these requirements, the
developer has a list of customers who each desire different requirements for
their own needs, with each customer having a weight given to them to indicate
their important to the developer.
% section overview (end)

\section{Problem Representation}
\label{sec:problem_representation}
\subsection{The Input File} % (fold)
\label{sub:the_input_file}
The data for requirements and customers were stored in a file made up of three
segments:

\begin{enumerate}
    \item The number of requirements and their cost.
    \item The dependencies between requirements.
    \item The customers, their weight to the company and their own requirements.
\end{enumerate}

Segment 1 starts by declaring the levels of requirements (ignored in evaluation
as per instructions), and is followed by two lines per level: the number of
requirements at said level in one line, and the proceeding line listing the cost
of each requirement. The position of the cost in this line indicates the ID of
the requirement.

Segment 2 starts by declaring the number of dependencies \emph{d}, with the
proceeding \emph{d} lines being pairs of IDs specifying these dependencies.
Again, instructions specify that these can be ignored.

Finally, Segment 3 again starts by declaring the number of customers \emph{c},
with the next \emph{c} lines listing the weight, number of requirements and then
those requirements.
% subsection the_input_file (end)

\subsection{Storing Data} % (fold)
\label{sub:storing_data}
The problem requires maintaining two collections: the requirements and the
customers. To represent these, two classes \emph{Requirement} and
\emph{Customer} were used to contain the required information after being parsed
from the input file.

The \emph{Requirement} class contained the level, cost and list of the IDs of
requirements it depends on. The \emph{Customer} class contains a list of IDs of
requirements it wants, and the customer's weight.
% subsection storing_data (end)

\subsection{Solution Representation} % (fold)
\label{sub:Solution_representation}
As part of the Opt4J framework, a decision had to be made on representing a
potential solution as a genotype and phenotype as required for the framework.

I decided that, given a total of \emph{n} requirements, a \emph{BooleanGenotype}
would be the best genotype representation of a solution. Each genotype would
generate \emph{n} boolean values to indicate whether the requirement is present
or not in the solution, meaning it is very simple to convert into the phenotype.

For decoding genotypes into phenotypes, the use of a \emph{BooleanGenotype} made
the decoding very straight forward, since each boolean could be converted into a
`0' or '1' depending on the value.
% subsection genotype_representation (end)

\subsection{Fitness Functions} % (fold)
\label{sub:fitness_functions}
As part of evaluating a solution, in both a multi-objective and single-objective
optimisation, the two important factors for evaluating had to be defined: cost
and score.

\subsubsection{Cost} % (fold)
\label{ssub:cost}
The cost is simply the difference between a summation of each present
requirement's cost and the given budget of the problem space. The budget is
calculated by calculating the sum of all requirements' costs, and then
multiplying by a specified `cost ratio' value. By calculating this difference,
tolerance is allowed for being slightly over-budget or under-budget if it yields
a sufficiently improved score.

For the purposes of multi-objective optimisation, the cost should be minimised
to achieve an optimal solution.
% subsubsection cost (end)

\subsubsection{Score} % (fold)
\label{ssub:score}
Calculating score is slightly more complex. For every present requirement, all
customers who desire that requirement must be obtained. For each of these
customers, the value of that requirement must be calculated.

The value a customer \(c\) places on a requirement \(r\) is:
\[w * \frac{n - i}{1 + 2 + ... + n}\]
Where \(w\) is the weight of the customer, \(n\) is the number of requirements
of \(c\), and \(i\) is the 0-based index of \(r\) in \(c\)'s list.

For the purposes of multi-objective optimisation, the cost should be maximised
to achieve an optimal solution.
% subsubsection score (end)

\subsubsection{Single Objective} % (fold)
\label{ssub:single_objective}
The single-objective optimisation uses a weighted-sum approach using the fitness
functions defined above. The weight is determined by user input using Opt4J, and
since we have a case where one objective is maximised and other is minimised,
the weighted-sum is tweaked to be:
\[F(x) = w \times score(x) - (1 - w) \times cost(x)\]
% subsubsection single_objective (end)
% subsection fitness_functions (end)
% section problem_representation (end)

\section{Implementation}
\label{sec:implementation}
Implementation details here.
% section implementation (end)


\section{Comparison}
\label{sec:comparison}

\subsection{Configuration} % (fold)
\label{sub:configuration}
Configuration that lead to results.
% subsection configuration (end)

\subsection{Results} % (fold)
\label{sub:results}
Results from Opt4J.
% subsection results (end)

\subsection{Analysis} % (fold)
\label{sub:analysis}
Analysis of results.
% subsection analysis (end)
% section comparison (end)

% \bibliographystyle{plain}
% \bibliography{report} 
\end{document}